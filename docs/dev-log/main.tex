% !TeX program = xelatex
\documentclass[UTF8,cs4size,12pt,a4paper,hyperref]{ctexart}
\usepackage[top=2.5cm, 
            bottom=2.5cm,
            left=1.5cm,
            right=1.5cm,
            twoside,
            headheight=12mm]
            {geometry}
\usepackage[super,comma,square,sort&compress]{natbib}
\CTEXoptions[bibname={\bfseries\zihao{-4}Refenrece}]
\linespread{1.6667} %文字间行距为24pt
\usepackage{multirow}
\usepackage{chemfig}
\usepackage{mhchem} % 化学式相关
\usepackage{amssymb}
\usepackage[authormarkup=none,
            xcolor={dvipdf,gray}]{changes} % 修订相关
%\usepackage[final]{changes} %禁用修订
\definechangesauthor[name={zbwu}, color=blue]{wu}
\setcommentmarkup{{\IfIsColored{\color{authorcolor}}{}(#1)}}
% \setcommentmarkup{(#1)}
\usepackage[para]{threeparttable}
\usepackage{setspace}
\usepackage{caption,subcaption}
\usepackage{bicaption}
\DeclareCaptionOption{english}[]{%
        \renewcommand\figurename{Figure}%
        \renewcommand\tablename{Table}}
% \captionsetup[bi-second]{english}
\captionsetup{font={small,stretch=1.25},labelfont=bf, justification=centering}
\captionsetup{labelsep=space}
\renewcommand{\thefigure}{\arabic{section}-\arabic{figure}}
\renewcommand{\thetable}{\arabic{section}-\arabic{table}}

\usepackage{booktabs}
\usepackage{graphics}
\usepackage{graphicx}
\counterwithin{figure}{section} %图片按章节编号1.x 2.x..
\usepackage{textcomp} % 摄氏度\textcelsius%打法
\usepackage{pdfpages}
\renewcommand{\arraystretch}{2} %表格数学公式
\usepackage{tabularx}
\newcolumntype{Y}{>{\centering\arraybackslash}X} %表格对齐
\usepackage{xcolor}
\colorlet{darkred}{red!50!black}
\definecolor{tiffany}{rgb}{0.505882, 0.847059, 0.815686}
\newcommand\diff{\, \mathrm{d}}
\usepackage{amsmath,mathrsfs}
\numberwithin{equation}{section} %公式按照章节编号1.x
\usepackage{braket}
%\usepackage{tabularx}
\setlength{\bibsep}{0pt} %设置文献之间的距离
\bibliographystyle{abbrv}
% \setmainfont{Times New Roman} %英文字体选择times new roman
% \graphicspath{{figure/}} 
\usepackage{fancyhdr}
\usepackage[colorlinks=true,urlcolor=blue,%
           linkcolor=red,
           pdfauthor={zbwu1996@gmail.com},
           pdftitle={Dev-log},
           pdfsubject={Dev-log},
           bookmarksnumbered=true]{hyperref}
\usepackage[capitalize]{cleveref} % 引用相关,能多个引用
 % 一次引用多个对象标签引用对象衔接
\newcommand{\crefpairconjunction}{~和~}
\newcommand{\crefmiddleconjunction}{、}
\newcommand{\creflastconjunction}{~和~}
\newcommand{\crefpairgroupconjunction}{~和~}
\newcommand{\crefmiddlegroupconjunction}{、}
\newcommand{\creflastgroupconjunction}{~和~}
\newcommand{\crefrangeconjunction}{~}
\crefname{figure}{图}{图}
\crefname{table}{表}{表}
\crefname{algorithm}{算法}{算法}
\crefname{theorm}{定理}{定理}
\crefname{definition}{定义}{定义}
\crefname{lemma}{引理}{引理}

% \usepackage{capption}
% \newcommand{\figref}[1]{图\ref{#1}}

\setlength{\belowcaptionskip}{3pt plus 1pt minus 2pt}

\usepackage{tocloft} %目录控制
    \renewcommand{\contentsname}{\hspace*{\fill}Contents\hspace*{\fill}}
\renewcommand\cftdot{…}
\renewcommand{\cftdotsep}{0} %引导点
\renewcommand{\cftsecleader}{\cftdotfill{\cftsecdotsep}}
\renewcommand{\cftsecdotsep}{\cftdotsep}
% TODO:
% \ctexset[name={,}]{section}
% \ctexset[number={\arabic{section}}]{section}
\CTEXsetup[name={,}]{section}
\CTEXsetup[number={\arabic{section}}]{section}
\usepackage{bookmark}

\usepackage{listings}
\usepackage{lstfiracode}
\lstset{
    basicstyle          =   \ttfamily,          % 基本代码风格
    keywordstyle        =   \bfseries,          % 关键字风格
    commentstyle        =   \rmfamily\itshape,  % 注释的风格,斜体
    stringstyle         =   \ttfamily,  % 字符串风格
    flexiblecolumns,                % 别问为什么,加上这个
    numbers             =   left,   % 行号的位置在左边
    showspaces          =   false,  % 是否显示空格,显示了有点乱,所以不现实了
    numberstyle         =   \zihao{-5}\ttfamily,    % 行号的样式,小五号,tt等宽字体
    showstringspaces    =   false,
    captionpos          =   t,      % 这段代码的名字所呈现的位置,t指的是top上面
    frame               =   lrtb,   % 显示边框
}

\lstdefinestyle{Python}{
    language        =   Python, % 语言选Python
    basicstyle      =   \zihao{-5}\ttfamily,
    numberstyle     =   \zihao{-5}\ttfamily,
    keywordstyle    =   \color{blue},
    % keywordstyle    =   [2] \color{teal},
    stringstyle     =   \color{magenta},
    commentstyle    =   \color{red}\ttfamily,
    breaklines      =   true,   % 自动换行,建议不要写太长的行
    columns         =   fixed,  % 如果不加这一句,字间距就不固定,很丑,必须加
    basewidth       =   0.30em,
}

\makeatletter
\renewcommand\normalsize{
  \@setfontsize\normalsize{12pt}{12pt} % 小四对应12pt
  \setlength\abovedisplayskip{4pt}
  \setlength\abovedisplayshortskip{4pt}
  \setlength\belowdisplayskip{\abovedisplayskip}
  \setlength\belowdisplayshortskip{\abovedisplayshortskip}
  \setlength{\baselineskip}{20pt} % 设置固定行间距为20pt
  \let\@listi\@listI}
  
\def\defaultfont{\renewcommand{\baselinestretch}{1.0}\normalsize\selectfont}
% 设置行距和段落间垂直距离
\renewcommand{\CJKglue}{\hskip -0.08pt plus 0.08\baselineskip} % 每行大概35个字符

\makeatother
\renewcommand{\sectionmark}[1]{\markright{\thesection-\ #1}}
\begin{document}
% \includepdf[pages={1,2}]{cover.pdf}
\pagenumbering{Roman}
\phantomsection
\addcontentsline{toc}{section}{Contents}
\tableofcontents
\clearpage
% \cleardoublepage
\pagestyle{fancy} % 使用fancy 风格
\fancyhf{} % 清除所有页眉页脚
\cfoot{\thepage} % 页脚居中页码
\fancyhead[LO]{\zihao{5}\slshape \rightmark}
\fancyhead[RE]{\zihao{5}\href{https://github.com/1996wu}{zbwu1996@gmail.com}}
% \fancyhead[RO,LE]{\includegraphics[height=10mm]{bnu-logo.pdf}}
\pagenumbering{arabic}
% !TeX root = ./main.tex
\section{ansatz}
\subsection{Recurrent Neural Network}
\begin{equation}
    \begin{split}
    \Psi(\mathbf{x}) = \sum_{\mathbf{x}}\psi(\mathbf{x})\ket{\mathbf{x}} &= \sum_{\mathbf{x}}\sqrt{P(\mathbf{x})}\ket{\mathbf{x}} \quad \text{Real} \\
    \text{or} & = \sum_{\mathbf{x}}\exp{[i\phi(\mathbf{x})]}\sqrt{P(\mathbf{x})}\ket{\mathbf{x}} \quad \text{Complex}
    \end{split}
\end{equation}
Conditional probability\cite{HibatAllah2020} $P_i$:
\begin{equation}
    \begin{split}
    P_i& = y_i^{(1)} \cdot x_i \\
    y_i^{(1)} & = \mathbf{softmax}(\underbrace{U^{(1)}\mathbf{h}_i + \mathbf{c}^{(1)}}_{\text{Linear Layers-1}}) \\
    \mathbf{softmax}(v_n) & = \frac{\exp{(v_n)}}{\sum_i \exp{(v_i)}}
    \end{split}
\end{equation}
phase:
\begin{equation}
    \begin{split}
    \phi_i & = y_i^{(2)} \cdot x_i \\
    y_i^{(2)} & = \pi \mathbf{softsign}(\underbrace{U^{(2)}\mathbf{h}_i + \mathbf{c}^{(2)}}_{\text{Linear Layers-2}}) \\
    \mathbf{softsign}(x) & = \frac{x}{1 + x} \in (-1, 1)
    \end{split}
\end{equation}
这里Linear Layers-1和Linear Layers-2可以共用同一套参数.\\
$x_i$是\textbf{ont-hot encoding}, 对于spin $\frac{1}{2}$体系:
\begin{equation}
    x_i \in (1, 0) \ or \ (0, 1)
\end{equation}
对于有\textit{N}个轨道的体系有
\begin{equation}
    \begin{split}
    P(\mathbf{x}) & = \prod_{i=1}^N P_i\\
    \phi(\mathbf{x}) &= \sum_{i=1}^N \phi_i
    \end{split}
\end{equation}

\subsection{Restricted Boltzmann Machine}
\begin{equation}
    \begin{split}
    \psi_{\theta}(\mathbf{x}) & = \textcolor{teal}{\exp}{\sum_{j=1}^{N_v}a_jx_j} \times 
        \prod_i^{N_h}\textcolor{violet}{2\cosh}(b_i + \sum_{j=1}^{N_v}W_{ij}x_j) \\
        \text{or} & = \prod_i^{N_h}\textcolor{violet}{2\cos}(b_i + \sum_{j=1}^{N_v}W_{ij}x_j) \quad 
        \textbf{cos-type}\\
        \text{or} & = \textcolor{teal}{\tanh}{\sum_{j=1}^{N_v}a_jx_j} \times 
        \prod_i^{N_h}\textcolor{violet}{2\cosh}(b_i + \sum_{j=1}^{N_v}W_{ij}x_j) \quad
        \textbf{tanh-type}
    \end{split}
\end{equation}
where $x_j$ represents the visible spin, $h_i$ the hidden spin and $W_{ij}$ weight parameters.
$N_v$, $N_h$ are the number of visible , hidden spins and weight respectively.\\
default:
\begin{equation}
    \alpha = N_h/N_v
\end{equation}

\subsubsection{AR-RBM}
\textit{K}-th sites\ ($0 < K < n_{sorb}=N_v$), $\mathbf{W}_K : (N_h, K)$\cite{bortone2023impact}
\begin{equation}
    \begin{split}
    \mathbf{x}_{K1} & = (x_0, \cdots, x_{K-1}, 1) \\
    \mathbf{x}_{K2} & = (x_0, \cdots, x_{K-1}, -1) \\
    \end{split}
\end{equation}

\begin{equation}
    \begin{split}
    \psi_{\theta}(\mathbf{x}_{K1}) & = \prod_{i}^{N_h}\textcolor{violet}{2\cos}
    (b_i + \sum_{\textcolor{blue}{k}=1}^{\textcolor{blue}{K}}W_{i\textcolor{blue}{k}}x_{\textcolor{blue}{k}}) \\
    \psi_{\theta}(\mathbf{x}_{K2}) & = \prod_{i}^{N_h}\textcolor{violet}{2\cos}
    (b_i + \sum_{\textcolor{blue}{k}=1}^{\textcolor{blue}{K}}W_{i\textcolor{blue}{k}}x_{\textcolor{blue}{k}}) \\
    \Theta & = (b_i + \sum_{\textcolor{blue}{k}=1}^{\textcolor{blue}{K}}W_{i\textcolor{blue}{k}}x_{\textcolor{blue}{k}})
    \end{split}
\end{equation}
Normalized $\psi_{\theta}(\mathbf{x}_{K1}), \psi_{\theta}(\mathbf{x}_{K2})$
to $\widetilde{\psi}_{\theta}(\mathbf{x}_{K1}), \widetilde{\psi}_{\theta}(\mathbf{x}_{K2})$

\begin{equation}
    \psi_{\theta}(\mathbf{x}) = \prod_{i=1}^{N_v}\frac{\widetilde{\psi}_i(x_i|\mathbf{x}_{<i})}
    {\sqrt{\sum_{x^\prime = 0}^{D-1}\vert \widetilde{\psi}_i(x^\prime |\mathbf{x}_{<i})\vert^2}}
    \enskip \mathrm{s.t.} ~ D = 2
\end{equation}

\begin{figure}[htp]
    \centering
    \includegraphics[width=0.5\textwidth]{AR-RBM.pdf}
    \caption{AR-RBM测试:\ce{H4}-1.40, $\alpha = 2, \mathrm{random\ seed} = 112123, AE= 0.00001$}
\end{figure}

\subsection{Transformer}
\textcolor{darkred}{\textbf{TODO:}} ref:\cite{zhang2023transformer,wu2023nnqs}
% !TeX root = ./main.tex
\section{distributed-train}
\subsection{distributed}
\noindent 分布式并行VMC大致流程:\\
\textbf{方案1}-每个rank独立采样计算:
\begin{enumerate}
    \item 每个rank独立采样,并去重(sample-unique, sample-counts)
    \item 每个rank 独立计算$eloc$
    \item \textbf{All-Reduce} 所有$eloc$得到平均值$eloc_{mean}$
    \item 计算loss function: $loss = 2\Re\left[\braket{\ln\psi^* eloc } -\braket{\ln\psi^*}\braket{eloc}\right]$
    \item loss.backword(), \textbf{DDP}自动\textbf{All-Reduce}平均梯度,保证每个rank梯度一致.
\end{enumerate}
\textbf{方案2}-每个rank独立采样, 通讯划分batch 计算$eloc$:
\begin{enumerate}
    \item 每个rank独立采样,并去重(sample-unique, sample-counts)
    \item \textcolor{darkred}{\textbf{Gather} 每个rank 的unique, counts.}
    \item \textcolor{darkred}{合并所有unique, counts, 计算prob.}
    \item \textcolor{darkred}{\textbf{Scatter} unique, counts, prob 到每个rank$(prob = prob * \mathrm{word-size})$.}
    \item 剩下流程与方案一相同.
\end{enumerate}
\begin{equation}
    \begin{split}
        loss =& 2\Re\left[\sum_{i=1}^N \ln{(\psi_i^*)} p_i eloc_i -
            \sum_{i=1}^N p_i eloc_i \sum_{i=1}^{N}\ln{(\psi_i^*)}p_i\right] \\
        = &2 \Re\left[\sum_{i=1}^N \ln{(\psi_i^*)} p_i eloc_i -
            eloc_{\mathrm{mean}} \sum_{i=1}^{N}\ln{(\psi_i^*)}p_i\right] \\
        = &2 \Re\left[\sum_{i=1}^{\textcolor{red}{n_0}} \ln{(\psi_i^*)} p_i eloc_i - 
            eloc_{\mathrm{mean}} \sum_{i=1}^{\textcolor{red}{n_0}}\ln{(\psi_i^*)}p_i\right]
             \quad \textcolor{red}{\mathrm{rank_0}} \\
        + & \textcolor{teal}{\cdots}\textnormal{(每个rank 计算不同batch)} \\
        + & 2\left[\sum_{i=1}^{\textcolor{green}{n_p}} \ln{(\psi_i^*)} p_i eloc_i -
            eloc_{\mathrm{mean}} \sum_{i=1}^{\textcolor{green}{n_p}}\ln{(\psi_i^*)}p_i\right] 
            \quad \textcolor{green}{\mathrm{rank_p}}\\
        &\mathrm{ s.t. } \sum_{l=0}^{p}n_l = N
    \end{split}
\end{equation}
\subsection{pre-train}

使用UCISD波函数$\ket{\psi_{ci}}$进行pre-train,与QGT(Quantum Geometric Tensor)类似, 定义$ovlp$:
\begin{equation}
    \begin{split}
        ovlp & = \frac{\braket{\psi|\psi_{ci}}{\braket{\psi_{ci}|\psi}}}
        {\braket{\psi|\psi}\braket{\psi_{ci}|\psi_{ci}}} \\
        & = \frac{\sum_n\braket{\psi|n}\braket{n|\psi_{ci}}\braket{\psi_{ci}|\psi}}
        {\sum_n{\braket{\psi|n}\braket{n|\psi}}} \\
        & = \frac{\sum_n\braket{\psi|n}\braket{n|\psi}\braket{n|\psi_{ci}}\frac{\braket{\psi_{ci}|\psi}}{\braket{n|\psi}}}
        {\sum_n \Vert \braket{\psi|n} \Vert^2} \\
        & = \frac{\textcolor{teal}{\sum_n \Vert \braket{\psi|n} \Vert^2} \braket{n|\psi_{ci}}\frac{\braket{\psi_{ci}|\psi}}{\braket{n|\psi}}}
        {\textcolor{teal}{\sum_n \Vert \braket{\psi|n} \Vert^2}} \\
        & = \textcolor{teal}{p(n)}\textcolor{violet}{\braket{n|\psi_{ci}}\frac{\braket{\psi_{ci}|\psi}}{\braket{n|\psi}}} \\
        & = \textcolor{teal}{p(n)}\textcolor{violet}{ovlp_{local}} \\
        & = \mathbb{E}_p\left[ \textcolor{violet}{\braket{n|\psi_{ci}}\frac{\braket{\psi_{ci}|\psi}}{\braket{n|\psi}}} \right] \\
    \end{split}
\end{equation}
量子几何张量QGT:
\begin{equation}
    \gamma(\psi, \phi) = \arccos \sqrt{\frac{\braket{\psi|\phi}\braket{\phi|\psi}}{\braket{\psi|\psi}\braket{\phi|\phi}}}
\end{equation}
这里$\ket{\psi_{ci}}$为pre-train的波函数,
$p(n)$来自之前的sampling,只计算与pre-ci重叠的部分.
在计算pre-ci和sampling中重复部分时,由于之前sampling过程中,会经历\textbf{Gather}-\textbf{merge}-\textbf{Scatter},
merge会使用\textit{torch.unique}函数,涉及到\textbf{onstate排序}.
因此不同rank可能会出现重叠为\textbf{0}的情况.
如果采用分布式,需要\textbf{All-Reduce}得到$ovlp_\mathrm{mean}$.
\begin{lstlisting}[language=Python]
nbatch = psi_sample.size(0)
oloc = torch.zeros(nbatch, dtype=self.dtype, device=self.device)
model_psi = self.model(self.ci_state).to(self.dtype)  # <n|psi_ci>
psi0 = torch.dot(self.pre_ci_coeff.conj(), model_psi)  # <psi_ci|psi>
# <n|psi_ci><psi_ci|psi>/<n|psi>
oloc_nonzero = -1 * self.pre_ci_coeff[idx_ci] * psi0 / psi_sample[idx_sample]
# ovlp part is non-zero, other part is zeros
oloc[idx_sample] = oloc_nonzero
ovlp = torch.dot(oloc_nonzero, prob[idx_sample])
# idx_sample/idx_ci 为pre-ci和sampling重叠部分的索引
\end{lstlisting}
$\textcolor{violet}{\ket{\psi_{ci}}\bra{\psi_{ci}}}\Rightarrow H $,与VMC的梯度类似, 梯度为:
\begin{equation}
    \nabla_{\theta}ovlp = -2\Re\left[\braket{\nabla_{\theta}\ln(\psi_{\theta}^*) \times ovlp_{local}}
                        -\braket{\nabla_{\theta}\ln(\psi_{\theta}^*)}\braket{ovlp_{local}}\right]
\end{equation}

另外一种pre-train方法,不考虑sampling, 用最小二乘法(Least Sqaure Method)拟合$\ket{\psi_{ci}}$,
该方法目前\textbf{不支持分布式}运行(归一化ci系数时,分布式暂时无法实现带梯度的Gather),每个rank的loss都相同.
\begin{lstlisting}[language=Python]
psi = self.model(self.ci_state.requires_grad_())
model_CI = psi / torch.norm(psi).flatten().to(self.dtype)
ovlp = torch.einsum("i, i", model_CI, self.pre_ci_coeff)
loss = 1 - ovlp.norm() ** 2
\end{lstlisting}
使用最小二乘法拟合\added[id=wu]{和\textit{ovlp}时},
注意拟合能量和pre-train$\ket{\psi_{ci}}$能量.
\deleted[id=wu,comment={采样的概率也是归一化的}]{如果使用\textit{ovlp}, 则能量参考无意义(???).}
\highlight[id=wu, comment={Hilbert-space实现}]{\textit{ovlp}计算能量, \textit{ci}系数是投影到CI-space or Hilbert-space?}
\begin{equation}
    e = \frac{\braket{\psi|H|\psi}}{\braket{\psi|\psi}} = \sum_{ij}c_i\braket{i|H|j}c_j^*
    \label{ci}
\end{equation}

\begin{lstlisting}[language=Python]
hij = get_hij_torch(onstate, onstate, h1e, h2e, sorb, nele).type_as(coeff)
e = torch.einsum("i, ij, j", coeff.flatten(), hij, coeff.flatten().conj()) + ecore
\end{lstlisting}

\begin{figure}[htp]
    \begin{subfigure}[b]{0.48\textwidth}
        \centering
        \includegraphics[width=1.0\textwidth]{H10-2.00-1-pre-train.pdf}
        \caption{$ovlp$拟合}
    \end{subfigure}
    \begin{subfigure}[b]{0.48\textwidth}
        \centering
        \includegraphics[width=1.0\textwidth]{H10-2.00-pre-train.pdf}
        \caption{最小二乘法拟合}
    \end{subfigure}
    \caption{\ce{H_10}-2.00两种不同pre-train方法}
\end{figure}

\crefrange{H6-1.60-UCISD-sample-1}{H6-1.60-UCISD-exact-2}为\ce{H_6}-1.60 pre-train测试结果,
随机数种子:$111\ 222\ 333\ 444\ 555$,\cref{H6-1.60-UCISD-sample-1,H6-1.60-UCISD-sample-2}为采样pre-train,
$n_\mathrm{sample}=500000/rank$,
\cref{H6-1.60-UCISD-exact-1,H6-1.60-UCISD-exact-2}为精确优化pre-train.\\
\indent \cref{H6-1.60-UCISD-sample-2}中random-seed=333预训练完成后, 通过\cref{ci}计算能量为\textbf{-2.44091829},其中$c_i$来自采样概率. 
若$c_i$为$\ket{\psi}$在CISD-space正交归一化, 能量为\textbf{-2.85258098},
在FCI-space空间正交归一化, 能量为\textbf{-2.83488588}.
VMC算得初始能量(采样)$e=-2.836264472$,接近$e_{UCISD}$.
变分优化2000步后,$e=-2.658145314$,HF能量$e_{HF}=-2.664983075$.\textbf{能量上升},这不符合变分优化原理.


\begin{table}[htp]
\centering
\captionsetup{labelfont=bf,skip=0pt,font=small,labelsep=space}  
\caption{\ce{H_6}-1.60使用UCISD波函数pre-train\tnote{d}结果,\textit{ovlp}和{能量}.}
% \renewcommand{\arraystretch}{1.25}
\begin{threeparttable}
\begin{tabular}{c|*{4}{p{6em}|}} 
\toprule
&\multicolumn{2}{c|}{$n_{param}=522$\tnote{a}} & \multicolumn{2}{c|}{$n_{param}=1922${\tnote{b}}}\\
\hline
random seed & sampling\tnote{c} & exact& sampling & exact\\
\hline 
\multirow{2}{*}{111} & 0.7034 & 0.7035 & 0.7008 & 0.7389  \\
& -2.65297524 & -2.65197247& -2.66497836 & -2.60799245\\
\hline
\multirow{2}{*}{222} & 0.7985 & 0.8653 & 0.7634 & 0.7591 \\
 & -2.58765390 & -2.52685629 & -2.64780052 & -2.64581788 \\
\hline
\multirow{2}{*}{333} & 0.7648 & 0.7731 & \textcolor{blue}{\textbf{0.9457}} & \textcolor{blue}{\textbf{0.9500}} \\
 & -2.64793674 & -2.64738522 & \textcolor{darkred}{\textbf{-2.44091829}} & \textcolor{darkred}{\textbf{-2.4548528}}\\
\hline
\multirow{2}{*}{444} & 0.7423 & \textcolor{blue}{\textbf{0.8506}}   & 0.7974 & 0.7822 \\
& -2.58655099 & \textcolor{darkred}{\textbf{-2.43728032}} & -2.59276192 & -2.59365129\\
\hline
\multirow{2}{*}{555} & 0.7287 & 0.7031 & 0.7606 & \textcolor{blue}{\textbf{0.9078}} \\
& -2.63931995 & -2.65291995 & -2.61340527 & \textcolor{darkred}{\textbf{-2.44117917}} \\
\bottomrule
% \multicolumn{5}{l}{}\\
\end{tabular}
\begin{tablenotes}
\item[a,b] {RNN 隐藏层数目分别为8,16\\}
\item[c] {$n_{sample}:500000,e_{UCISD}=-2.89603126, e_{HF}=-2.664983075$\\}
\item[d] {opt-type: Adam, iteration-time: 5000, $lr=0.005, lr_{schedule}:(1 + step / 5000)^{-1}$}
\end{tablenotes} 
\end{threeparttable} 
\end{table}

\begin{figure}[htp]
    \centering
    \includegraphics[width=0.95\textwidth]{H6-1.60-UCISD-sample-1.pdf}
    \caption{\ce{H_6}-1.60使用UCISD波函数\textit{ovlp}\ pre-train,
            param$_{\mathrm{RNN}}=530$.}
    \label{H6-1.60-UCISD-sample-1}
\end{figure}

\begin{figure}[htp]
    \centering
    \includegraphics[width=0.95\textwidth]{H6-1.60-UCISD-sample-2.pdf}
    \caption{\ce{H_6}-1.60使用UCISD波函数\textit{ovlp}\ pre-train,
            param$_{\mathrm{RNN}}=1922$.
            \highlight[id=wu, comment={$e_{UCISD}=-2.89603126$}]
            {random-seed=333,$ovlp=0.945,\ e=-2.44091829$,能量计算有误}
            }
    \label{H6-1.60-UCISD-sample-2}
\end{figure}

\begin{figure}[htp]
    \centering
    \includegraphics[width=0.95\textwidth]{H6-1.60-UCISD-exact-1.pdf}
    \caption{\ce{H_6}-1.60使用UCISD波函数精确优化\textit{ovlp}\ pre-train,
            param$_{\mathrm{RNN}}=530$.}
    \label{H6-1.60-UCISD-exact-1}
\end{figure}

\begin{figure}[htp]
    \centering
    \includegraphics[width=0.95\textwidth]{H6-1.60-UCISD-exact-2.pdf}
    \caption{\ce{H_6}-1.60使用UCISD波函数精确优化\textit{ovlp}\ pre-train,
            param$_{\mathrm{RNN}}=1922$.
            \highlight[id=wu]
            {random-seed=333,$ovlp=0.949,\ e=-2.45485280$,能量计算有误}}
    \label{H6-1.60-UCISD-exact-2}
\end{figure}
\bibliography{cite.bib}
\addcontentsline{toc}{section}{Refenrece} 
\end{document}