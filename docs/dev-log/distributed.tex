% !TeX root = ./main.tex
\section{distributed-train}
\subsection{distributed}
\noindent 分布式并行VMC大致流程:\\
\textbf{方案1}-每个rank独立采样计算:
\begin{enumerate}
    \item 每个rank独立采样,并去重(sample-unique, smaple-counts)
    \item 每个rank 独立计算$eloc$
    \item \textbf{All-Reduce} 所有$eloc$得到平均值$eloc_{mean}$
    \item 计算loss function: $loss = 2\Re\left[\braket{\ln\psi^* eloc } -\braket{\ln\psi^*}\braket{eloc}\right]$
    \item loss.backword(), \textbf{DDP}自动\textbf{All-Reduce}平均梯度,保证每个rank梯度一致.
\end{enumerate}
\textbf{方案2}-每个rank独立采样, 通讯划分batch 计算$eloc$:
\begin{enumerate}
    \item 每个rank独立采样,并去重(sample-unique, smaple-counts)
    \item \textcolor{darkred}{\textbf{Gather} 每个rank 的unique, counts.}
    \item \textcolor{darkred}{合并所有unique, counts, 计算prob.}
    \item \textcolor{darkred}{\textbf{Scatter} unique, counts, prob 到每个rank$(prob = prob * \mathrm{word-size})$.}
    \item 剩下流程与方案一相同.
\end{enumerate}
\begin{equation}
    \begin{split}
        loss =& 2\Re\left[\sum_{i=1}^N \ln{\psi_i} p_i eloc_i - \sum_{i=1}^N p_i eloc_i \sum_{i=1}^{N}\ln{\psi_i}p_i\right] \\
        = &2 \Re\left[\sum_{i=1}^N \ln{\psi_i} p_i eloc_i - eloc_{\mathrm{mean}} \sum_{i=1}^{N}\ln{\psi_i}p_i\right] \\
        = &2 \Re\left[\sum_{i=1}^{\textcolor{red}{n_0}} \ln{\psi_i} p_i eloc_i - eloc_{\mathrm{mean}} \sum_{i=1}^{\textcolor{red}{n_0}}\ln{\psi_i}p_i\right] \quad \textcolor{red}{\mathrm{rank_0}} \\
        + & \textcolor{teal}{\cdots}\textnormal{(每个rank 计算不同batch)} \\
        + & 2\left[\sum_{i=1}^{\textcolor{green}{n_p}} \ln{\psi_i} p_i eloc_i - eloc_{\mathrm{mean}} \sum_{i=1}^{\textcolor{green}{n_p}}\ln{\psi_i}p_i\right] \quad \textcolor{green}{\mathrm{rank_p}}\\
        &\mathrm{ s.t. } \sum_{l=0}^{p}n_l = N
    \end{split}
\end{equation}
\subsection{pre-train}

使用UCISD波函数$\ket{\psi_{ci}}$进行pre-train,与QGT(Quantum Geometric Tensor)类似, 定义$ovlp$:
\begin{equation}
    \begin{split}
        ovlp & = \frac{\braket{\psi|\psi_{ci}}{\braket{\psi_{ci}|\psi}}}
        {\braket{\psi|\psi}\braket{\psi_{ci}|\psi_{ci}}} \\
        & = \frac{\sum_n\braket{\psi|n}\braket{n|\psi_{ci}}\braket{\psi_{ci}|\psi}}
        {\sum_n{\braket{\psi|n}\braket{n|\psi}}} \\
        & = \frac{\sum_n\braket{\psi|n}\braket{n|\psi}\braket{n|\psi_{ci}}\frac{\braket{\psi_{ci}|\psi}}{\braket{n|\psi}}}
        {\sum_n \Vert \braket{\psi|n} \Vert^2} \\
        & = \frac{\textcolor{teal}{\sum_n \Vert \braket{\psi|n} \Vert^2} \braket{n|\psi_{ci}}\frac{\braket{\psi_{ci}|\psi}}{\braket{n|\psi}}}
        {\textcolor{teal}{\sum_n \Vert \braket{\psi|n} \Vert^2}} \\
        & = \textcolor{teal}{p(n)}\textcolor{violet}{\braket{n|\psi_{ci}}\frac{\braket{\psi_{ci}|\psi}}{\braket{n|\psi}}} \\
        & = \textcolor{teal}{p(n)}\textcolor{violet}{ovlp_{local}} \\
        & = \mathbb{E}_p\left[ \textcolor{violet}{\braket{n|\psi_{ci}}\frac{\braket{\psi_{ci}|\psi}}{\braket{n|\psi}}} \right] \\
    \end{split}
\end{equation}
量子几何张量QGT:
\begin{equation}
    \gamma(\psi, \phi) = \arccos \sqrt{\frac{\braket{\psi|\phi}\braket{\phi|\psi}}{\braket{\psi|\psi}\braket{\phi|\phi}}}
\end{equation}
这里$\ket{\psi_{ci}}$为pre-train的波函数,
$p(n)$来自之前的sampling,只计算与pre-ci重叠的部分.
在计算pre-ci和sampling中重复部分时,由于之前sampling过程中,会经历\textbf{Gather}-\textbf{merge}-\textbf{Scatter},
merge会使用\textit{torch.unique}函数,涉及到\textbf{onstate排序}.
因此不同rank可能会出现重叠为\textbf{0}的情况.
如果采用分布式,需要\textbf{All-Reduce}得到$ovlp_\mathrm{mean}$.
\begin{lstlisting}[language=Python]
model_ci = self.model(self.ci_state).to(self.dtype)
psi0 = torch.dot(self.pre_ci_coeff.conj(), model_ci)
ovlp_local = -1 * self.pre_ci_coeff[idx_ci] * psi0/psi_sample[idx_sample]
ovlp = torch.dot(ovlp_local, prob[idx_sample])
# idx_sample/idx_ci 为pre-ci和sampling重叠部分的索引
\end{lstlisting}
$\textcolor{violet}{\ket{\psi_{ci}}\bra{\psi_{ci}}}\Rightarrow H $,与VMC的梯度类似, loss function为:
\begin{equation}
    loss = -2\Re\left[\braket{\ln\psi^* \times ovlp_{local} } -\braket{\ln\psi^*}\braket{ovlp_{local}}\right]
\end{equation}

另外一种pre-train方法,不考虑sampling, 用最小二乘法(Least Sqaure Method)拟合$\ket{\psi_{ci}}$,
该方法目前\textbf{不支持分布式}运行(归一化ci系数时,分布式暂时无法实现带梯度的Gather),每个rank的loss都相同.
\begin{lstlisting}[language=Python]
psi = self.model(self.ci_state.requires_grad_())
model_CI = psi / torch.norm(psi).flatten().to(self.dtype)
ovlp = torch.einsum("i, i", model_CI, self.pre_ci_coeff)
loss = 1 - ovlp.norm() ** 2
\end{lstlisting}
使用最小二乘法拟合\added[id=wu]{和\textit{ovlp}时},
注意拟合能量和pre-train$\ket{\psi_{ci}}$能量.
\deleted[id=wu,comment={采样的概率也是归一化的}]{如果使用\textit{ovlp}, 则能量参考无意义(???).}
\highlight[id=wu, comment={Hilbert-space实现}]{\textit{ovlp}计算能量, \textit{ci}系数是投影到CI-space or Hilbert-space?}
\begin{equation}
    e = \frac{\braket{\psi|H|\psi}}{\braket{\psi|\psi}} = \sum_{ij}c_i\braket{i|H|j}c_j^*
    \label{ci}
\end{equation}

\begin{lstlisting}[language=Python]
hij = get_hij_torch(onstate, onstate, h1e, h2e, sorb, nele).type_as(coeff)
e = torch.einsum("i, ij, j", coeff.flatten(), hij, coeff.flatten().conj()) + ecore
\end{lstlisting}

\begin{figure}[htp]
    \begin{subfigure}[b]{0.48\textwidth}
        \centering
        \includegraphics[width=1.0\textwidth]{H10-2.00-1-pre-train.pdf}
        \caption{$ovlp$拟合}
    \end{subfigure}
    \begin{subfigure}[b]{0.48\textwidth}
        \centering
        \includegraphics[width=1.0\textwidth]{H10-2.00-pre-train.pdf}
        \caption{最小二乘法拟合}
    \end{subfigure}
    \caption{\ce{H_10}-2.00两种不同pre-train方法}
\end{figure}

\crefrange{H6-1.60-UCISD-sample-1}{H6-1.60-UCISD-exact-2}为\ce{H_6}-1.60 pre-train测试结果,
随机数种子:$111\ 222\ 333\ 444\ 555$,\cref{H6-1.60-UCISD-sample-1,H6-1.60-UCISD-sample-2}为采样pre-train,
$n_\mathrm{sample}=500000/rank$,
\cref{H6-1.60-UCISD-exact-1,H6-1.60-UCISD-exact-2}为精确优化pre-train.\\
\indent \cref{H6-1.60-UCISD-sample-2}中random-seed=333预训练完成后, 通过\cref{ci}计算能量为\textbf{2.44091829},其中$c_i$来自采样概率. 
若$c_i$为$\ket{\psi}$在CISD-space正交归一化, 能量为\textbf{-2.85258098},
在FCI-space空间正交归一化, 能量为\textbf{-2.83488588}.
VMC算得初始能量(采样)$e=-2.836264472$,接近$e_{UCISD}$.
变分优化2000步后,$e=-2.658145314$,HF能量$e_{HF}=-2.664983075$.\textbf{能量上升},这不符合变分优化原理.


\begin{table}[htp]
\centering
\captionsetup{labelfont=bf,skip=0pt,font=small,labelsep=space}  
\caption{\ce{H_6}-1.60使用UCISD波函数pre-train\tnote{d}结果,\textit{ovlp}和{能量}.}
% \renewcommand{\arraystretch}{1.25}
\begin{threeparttable}
\begin{tabular}{c|*{4}{p{6em}|}} 
\toprule
&\multicolumn{2}{c|}{$n_{param}=522$\tnote{a}} & \multicolumn{2}{c|}{$n_{param}=1922${\tnote{b}}}\\
\hline
random seed & sampling\tnote{c} & exact& sampling & exact\\
\hline 
\multirow{2}{*}{111} & 0.7034 & 0.7035 & 0.7008 & 0.7389  \\
& -2.65297524 & -2.65197247& -2.66497836 & -2.60799245\\
\hline
\multirow{2}{*}{222} & 0.7985 & 0.8653 & 0.7634 & 0.7591 \\
 & -2.58765390 & -2.52685629 & -2.64780052 & -2.64581788 \\
\hline
\multirow{2}{*}{333} & 0.7648 & 0.7731 & \textcolor{blue}{\textbf{0.9457}} & \textcolor{blue}{\textbf{0.9500}} \\
 & -2.64793674 & -2.64738522 & \textcolor{darkred}{\textbf{-2.44091829}} & \textcolor{darkred}{\textbf{-2.4548528}}\\
\hline
\multirow{2}{*}{444} & 0.7423 & \textcolor{blue}{\textbf{0.8506}}   & 0.7974 & 0.7822 \\
& -2.58655099 & \textcolor{darkred}{\textbf{-2.43728032}} & -2.59276192 & -2.59365129\\
\hline
\multirow{2}{*}{555} & 0.7287 & 0.7031 & 0.7606 & \textcolor{blue}{\textbf{0.9078}} \\
& -2.63931995 & -2.65291995 & -2.61340527 & \textcolor{darkred}{\textbf{-2.44117917}} \\
\bottomrule
% \multicolumn{5}{l}{}\\
\end{tabular}
\begin{tablenotes}
\item[a,b] {RNN 隐藏层数目分别为8,16\\}
\item[c] {$n_{sample}:500000,e_{UCISD}=-2.89603126, e_{HF}=-2.664983075$\\}
\item[d] {opt-type: Adam, iteration-time: 5000, $lr=0.005, lr_{schedule}:(1 + step / 5000)^{-1}$}
\end{tablenotes} 
\end{threeparttable} 
\end{table}

\begin{figure}[htp]
    \centering
    \includegraphics[width=0.95\textwidth]{H6-1.60-UCISD-sample-1.pdf}
    \caption{\ce{H_6}-1.60使用UCISD波函数\textit{ovlp}\ pre-train,
            param$_{\mathrm{RNN}}=530$.}
    \label{H6-1.60-UCISD-sample-1}
\end{figure}

\begin{figure}[htp]
    \centering
    \includegraphics[width=0.95\textwidth]{H6-1.60-UCISD-sample-2.pdf}
    \caption{\ce{H_6}-1.60使用UCISD波函数\textit{ovlp}\ pre-train,
            param$_{\mathrm{RNN}}=1922$.
            \highlight[id=wu, comment={$e_{UCISD}=-2.89603126$}]
            {random-seed=333,$ovlp=0.945,\ e=-2.44091829$,能量计算有误}
            }
    \label{H6-1.60-UCISD-sample-2}
\end{figure}

\begin{figure}[htp]
    \centering
    \includegraphics[width=0.95\textwidth]{H6-1.60-UCISD-exact-1.pdf}
    \caption{\ce{H_6}-1.60使用UCISD波函数精确优化\textit{ovlp}\ pre-train,
            param$_{\mathrm{RNN}}=530$.}
    \label{H6-1.60-UCISD-exact-1}
\end{figure}

\begin{figure}[htp]
    \centering
    \includegraphics[width=0.95\textwidth]{H6-1.60-UCISD-exact-2.pdf}
    \caption{\ce{H_6}-1.60使用UCISD波函数精确优化\textit{ovlp}\ pre-train,
            param$_{\mathrm{RNN}}=1922$.
            \highlight[id=wu]
            {random-seed=333,$ovlp=0.949,\ e=-2.45485280$,能量计算有误}}
    \label{H6-1.60-UCISD-exact-2}
\end{figure}